\chapter{Datos Escalares}

\section{N�meros}

Aunque los n�meros y las cadenas con frecuencia se pueden tratar como escalares, es �til observarlos inicialmente por separado. Vamos a ver primero los n�meros y luego pasamos a las cadenas.

\subsection{Todos los n�meros internamente tienen el mismo formato}

\label{Todos-los-n-meros-internamente-tienen-el-mismo-formato}

Como vera en los pr�ximos p�rrafos, se puede especif\mbox{}icar tanto n�meros enteros (como 255 o 2001) y n�meros de coma f\mbox{}lotante (n�meros reales con puntos decimales, como 3.1416 o 1.35 x 1025). Pero internamente, Perl calcula valores punto f\mbox{}lotantes de doble precisi�n. Esto signif\mbox{}ica que no hay valores enteros internamente en Perl. Una constante entera en Perl es tratada como su valor equivalente en coma f\mbox{}lotante. Probablemente no se dar� cuenta de la conversi�n, deje de buscar las distintas operaciones con enteros (en oposici�n a las operaciones de punto f\mbox{}lotante), porque simplemente no existen. \footnote{Existe un pragma llamado Integer, que permite realizar operaciones con enteros en lugar de punto f\mbox{}lotante, pero es otra cosa, y no es de lo que estamos hablando en este punto}

\subsection{Literales de punto f\mbox{}lotante}

\label{Literales-de-punto-flotante}

Un literal es la forma en que un valor se representa en el c�digo fuente en Perl. Un literal no es el resultado de una operaci�n de calculo o de una operaci�n de I/O. Es datos escritos directamente en el c�digo fuente.

Un literal de coma f\mbox{}lotante, ya debe serle familiar. N�meros con y sin punto f\mbox{}lotante tambi�n son permitidos (incluyendo el pref\mbox{}ijo opcional mas y menos), y de remate el indicador de notaci�n exponencial E.

\vspace{-6pt}
\small
\begin{Verbatim}[commandchars=\\\{\},numbers=left]
    1.25 
    255.000 
    255.0 
    7.25e45 
    
\end{Verbatim}
\vspace{-6pt}
\normalsize
